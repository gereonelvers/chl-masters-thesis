\chapter{\abstractname}

Despite growing interest in augmented reality (AR) for industrial applications, little is known about how people collaborate when sharing the same AR space. This thesis explores co-located human-human collaboration in industrial AR environments through an exploratory study, addressing three questions: how can AR collaboration be classified, what technical challenges emerge, and how do AR systems affect collaboration patterns and performance.

A systematic review of 32 studies identified four main collaboration types based on information flow and goals: expert support, creative workflows, precision tasks, and dynamic problem-solving. Common technical challenges include hardware limitations, spatial tracking problems, networking issues, and performance constraints. Based on the review, I developed a collaborative AR prototype using HoloLens 2 devices with marker-based tracking and custom networking that achieved sub-50ms response times for shared bridge building tasks.

An exploratory user study using this prototype with 16 participants tested four conditions: normal collaboration, silent collaboration, roleplay, and time pressure. Surprisingly, limiting communication appeared to improve performance rather than hurting it—pairs worked effectively together even without speaking. Partners synchronized their movement distances differently across conditions and developed consistent construction approaches that maintained quality regardless of how quickly they worked.

These preliminary findings suggest that common assumptions about communication needs in collaborative AR systems may warrant further investigation. Rather than maximizing communication options, well-designed constraints might help teams focus their efforts more effectively. This work offers initial insights into AR collaboration types and preliminary observations for designers considering collaborative AR systems for industrial use.

\textbf{Keywords:} augmented reality, human-human collaboration, industrial assembly, co-located collaboration, Design Science Research

\makeatletter
\ifthenelse{\pdf@strcmp{\languagename}{english}=0}
{\renewcommand{\abstractname}{Kurzfassung}}
{\renewcommand{\abstractname}{Abstract}}
\makeatother

\chapter{\abstractname ~(German)}

\begin{otherlanguage}{ngerman}
Trotz wachsenden Interesses an Augmented Reality (AR) für industrielle Anwendungen ist wenig darüber bekannt, wie Menschen zusammenarbeiten, wenn sie denselben AR-Raum teilen. Diese Arbeit erforscht die gemeinsame menschliche Zusammenarbeit in industriellen AR-Umgebungen durch eine explorative Studie und behandelt drei Fragen: Wie kann AR-Kollaboration klassifiziert werden, welche technischen Herausforderungen entstehen, und wie beeinflussen AR-Systeme Kollaborationsmuster und Leistung.

Eine systematische Übersicht von 32 Studien identifizierte vier Haupttypen der Zusammenarbeit basierend auf Informationsfluss und Zielen: Expertenunterstützung, kreative Arbeitsabläufe, Präzisionsaufgaben und dynamische Problemlösung. Häufige technische Herausforderungen umfassen Hardware-Limitationen, Probleme bei der räumlichen Verfolgung, Netzwerkprobleme und Leistungseinschränkungen. Basierend auf der Übersicht entwickelte ich einen kollaborativen AR-Prototyp mit HoloLens 2 Geräten, markerbasierter Verfolgung und angepasster Netzwerktechnik, der Antwortzeiten unter 50ms für gemeinsame Brückenbauaufgaben erreichte.

Eine explorative Benutzerstudie mit diesem Prototyp und 16 Teilnehmern testete vier Bedingungen: normale Zusammenarbeit, stille Zusammenarbeit, Rollenspiel und Zeitdruck. Überraschenderweise schien die Einschränkung der Kommunikation die Leistung zu verbessern, anstatt sie zu beeinträchtigen—Paare arbeiteten effektiv zusammen, auch ohne zu sprechen. Partner synchronisierten ihre Bewegungsdistanzen unterschiedlich über die Bedingungen hinweg und entwickelten konsistente Konstruktionsansätze, die die Qualität unabhängig von der Arbeitsgeschwindigkeit aufrechterhielten.

Diese Erkenntnisse deuten darauf hin, dass gängige Annahmen über Kommunikationsbedürfnisse in kollaborativen AR-Systemen weitere Untersuchungen rechtfertigen könnten. Anstatt Kommunikationsoptionen zu maximieren, könnten gut gestaltete Einschränkungen Teams dabei helfen, ihre Anstrengungen effektiver zu fokussieren. Diese explorative Arbeit bietet erste Einblicke in AR-Kollaborationstypen und Beobachtungen für Designer, die kollaborative AR-Systeme für industrielle Nutzung in Betracht ziehen.

\end{otherlanguage}


% Undo the name switch
\makeatletter
\ifthenelse{\pdf@strcmp{\languagename}{english}=0}
{\renewcommand{\abstractname}{Abstract}}
{\renewcommand{\abstractname}{Kurzfassung}}
\makeatother