\chapter{Introduction}\label{chapter:introduction }
\section{Motivation}
The advent of Industry 4.0 has transformed manufacturing paradigms, introducing unprecedented levels of automation, connectivity, and data-driven decision making \cite{moencks2022augmented}. However, this digital transformation has simultaneously revealed critical challenges in workforce development and human-machine collaboration. The increasing complexity of modern manufacturing systems, coupled with a rapidly aging workforce and persistent shortage of qualified personnel, has created an urgent need to provide adequate support to both existing and incoming labour \cite{moencks2022augmented, deSouza2020surveyIndustrialAR} \cite{ozkan2022analysing}. 


In that context, augmented reality (AR) has emerged as a transformative technology that can offer new ways in how workers interact with various industrial tasks such as manual assembly \cite{doerner2022xrTextbook}. By overlaying digital information on the physical surroundings of the user, the AR interface can provide real-time guidance to streamline task execution, reduce errors, and shorten training times \cite{wang2016arAssemblyLitRev}.
For example, AR has been recognised to benefit collaborative teaching, improve worker safety, and as a critical enabler of human-robot interaction \cite{Agati2020Augmented}.

However, despite these benefits and AR's potential to enhance collaboration, prior research has focused mainly on human-robot interaction (HRI) scenarios, reflecting industry's emphasis on automation and robotic integration \cite{Lukosch2015Collaboration}. Moreover, the existing examples of human-human collaboration typically investigated only remote or asynchronous collaboration \cite{rubart2022augmenting}. As a result, there remains little understanding of the unique collaboration dynamics that may arise in co-located human-human collaboration and the majority of technical implementations can be found in industry, unvalidated by empirical research. Co-located collaboration presents distinct challenges and opportunities compared to remote collaboration, including real-time spatial coordination, shared physical workspace management, and the potential for rich multimodal communication \cite{Lukosch2015Collaboration}.

\section{Objective}
Following the paradigm of Design Science Research\cite{hevner2007dsr}, I explore the applications of AR-enabled, co-located human-human collaboration on a joint industrial task. The Design Science Research approach is particularly well-suited to this investigation as it emphasizes the iterative development and evaluation of artefacts to address identified problems in real-world contexts \cite{hevner2007dsr}. Through a systematic literature review (SLR), exploratory implementation of a research artefact, and a user study conducted using the final artefact, I aim to answer the following research questions:

\subsection{Research Questions}
\begin{enumerate}
    \item[\textbf{RQ1:}] What types of human-human AR collaboration have been implemented in industrial applications, and how can these be classified based on their characteristics and use cases?
    
    \item[\textbf{RQ2:}] What are the challenges associated with different technical architectures for collaborative AR-based industrial task support, and how do they support multi-agent human interaction in joint task execution?
    
    \item[\textbf{RQ3:}] How does the implemented AR system influence collaboration dynamics, user satisfaction and performance during industrial task execution, particularly in relation to different task variants and different types of users (, e.\,g., varying personality traits)?
\end{enumerate}

\subsection{Research Approach}
Given the exploratory nature of this project, I follow the Design Science Research (DSR) paradigm \cite{hevner2007dsr}. The DSR approach is particularly well-suited for this investigation as it emphasizes the iterative development and evaluation of artefacts to address identified problems in real-world contexts. I structure the research according to DSR's three fundamental cycles:

\begin{itemize}
    \item \textbf{Relevance cycle:} Establishes the research context and requirements through systematic literature review and analysis of current industrial collaboration challenges.
    \item \textbf{Design cycle:} Involves the iterative development and refinement of the collaborative AR prototype, incorporating technical implementation decisions and usability considerations.
    \item \textbf{Rigor cycle:} Ensures scientific rigor through comprehensive evaluation using mixed-methods approaches, including quantitative performance metrics and qualitative user feedback.
\end{itemize}

I integrate three main phases in my research methodology: (1) systematic literature review following PRISMA guidelines to establish the current state of knowledge, (2) iterative artefact development using relevance, design, and rigor cycles, and (3) exploratory evaluation through a controlled user study employing mixed-methods data collection. This approach enables both initial theoretical insights and preliminary practical observations for collaborative AR system design and implementation.

\subsection{Research Contribution and Significance}
This exploratory research offers preliminary insights into both academic understanding and practical considerations for industrial AR collaboration. From a theoretical perspective, it begins to address the identified gap in understanding co-located human-human collaboration dynamics in AR-enhanced industrial settings. Through investigating collaboration patterns, communication mechanisms, and performance outcomes, this work provides initial contributions to the broader field of Computer-Supported Cooperative Work (CSCW) and human-computer interaction in industrial contexts.

The systematic literature review offers a classification framework for human-human AR collaboration in industrial applications, providing researchers and practitioners with a structured overview of current approaches and identifying areas for future investigation. The iterative development and evaluation of the research artefact provides initial insights into the technical challenges and design considerations for implementing collaborative AR systems in manufacturing environments.

Given the limited sample size (16 participants) and exploratory nature of this work, the majority of findings should be considered preliminary indicators rather than definitive conclusions. The research contributes to the growing body of knowledge on human factors in advanced manufacturing systems, while highlighting the need for larger-scale studies to validate these initial observations \cite{moencks2022augmented}.


\section{Structure}
Chapter \ref{chapter:background} provides relevant definitions, an overview of key concepts and related work from the three intersecting fields of AR, collaboration dynamics, and industrial applications.

Chapter \ref{chapter:literature} presents the systematic literature review (SLR), following the Preferred Reporting Items for Systematic Reviews and Meta-Analyses (PRISMA) \cite{page2021prisma} protocol, conducted to explore the current state of research in co-located human-human or hybrid collaboration using AR head-mounted displays (HMDs) in industrial applications.

Chapter \ref{chapter:artifact-implementation} describes the iterative development process of the collaborative AR prototype, including technical architecture decisions, implementation challenges, and solutions for networking, localization, and user interface design.

Chapter \ref{chapter:study-design} presents the design of the exploratory user study, based on the SLR findings, including participant sampling, study setup, procedures, and data collection instruments.

Chapter \ref{chapter:results} presents the results of the user study, which were collected using a mixed-methods approach, collecting both qualitative (, e.\,g., discussion with participants and their subjective feedback) and quantitative data (, e.\,g., task completion times and error rates).

Chapter \ref{chapter:discussion} discusses the results in the context of the research questions and existing literature, analysing the preliminary implications of the findings for AR-based collaboration in industrial settings.

Chapter \ref{chapter:conclusion-outlook} concludes the thesis by summarising the key insights, discussing the limitations of the study, and presenting recommendations for future research directions.