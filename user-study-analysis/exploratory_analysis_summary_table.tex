
\begin{table}[htbp]
\centering
\caption{Summary of Key Exploratory Analysis Findings}
\label{tab:exploratory_analysis_summary}
\begin{tabular}{@{}p{2.5cm}p{2.8cm}p{4.2cm}p{4.5cm}@{}}
\toprule
\textbf{Domain} & \textbf{Finding} & \textbf{Result} & \textbf{Implication} \\
\midrule
Task Variant Effects & Communication Constraints & Silent condition significantly faster (5.00 vs 10.72 min, p=0.025) & Communication overhead may reduce efficiency in co-located collaboration \\
 & Time Pressure & Timed variant 3x faster (3.08 vs 8.81 min) with no quality loss & Time constraints enhance focus without compromising outcomes \\
 & Split Objectives & Roleplay variant 2.6x slower with 2.2x more movement (p=0.031) & Role-based constraints create coordination overhead \\
Individual Differences & Personality Traits & All Big Five traits show weak correlations (|r| < 0.41, all p > 0.1) & Situational factors may dominate over personality traits \\
 & Experience Levels & Non-linear relationship; moderate experience shows worst performance & Intermediate experience may create problematic expectations \\
 & Learning Patterns & Limited experience users show largest SUS improvement (+9.6 points) & System design well-suited for AR/VR novices \\
Technical Architecture & Spatial Alignment & Movement asymmetry uncorrelated with performance (r=0.041, p=0.830) & Marker-based anchoring successfully eliminates spatial constraints \\
 & UI Design Impact & Significant SUS improvement from 67.5 to 72.5 across sessions & Spatially-anchored controls support learning and adaptation \\
 & Performance Consistency & High variability (CV=3.374) despite robust networking & Network robustness necessary but insufficient for consistency \\
\bottomrule
\end{tabular}
\end{table}
